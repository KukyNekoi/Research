\begin{figure}[!t]
\begin{tikzpicture}[
      start chain=1 going right,start chain=2 going right,node distance=-0.15mm
    ]  
    \node [on chain=1] at (-1.5,-.4) {A:};
    \foreach \x in {2,3,5,7,12,23,34,54,67,68,89} {
        \x, \node [draw,on chain=1] {\x};
    }    
    \node [on chain=2] at (-1.5,-1.0) {S:};
    \foreach \x in {11} {
        \x, \node [draw,on chain=2] {\x};
    } 
\end{tikzpicture}\\

\begin{tikzpicture}[
      start chain=1 going right,start chain=2 going right,node distance=-0.15mm
    ]  
    \node [on chain=1] at (-1.5,-.4) {A:};
    \foreach \x in {\textbf{2},3,5,7,12,23,34,54,67,68,89} {
        \x, \node [draw,on chain=1] {\x};
    }    
    \node [on chain=2] at (-1.5,-1.0) {S:};
    \foreach \x in {11} {
        \x, \node [draw,on chain=2] {\x};
    } 
\end{tikzpicture}\\

\begin{tikzpicture}[
      start chain=1 going right,start chain=2 going right,node distance=-0.15mm
    ]  
    \node [on chain=1] at (-1.5,-.4) {A:};
    \foreach \x in {\textbf{2},\textbf{3},5,7,12,23,34,54,67,68,89} {
        \x, \node [draw,on chain=1] {\x};
    }    
    \node [on chain=2] at (-1.5,-1.0) {S:};
    \foreach \x in {11} {
        \x, \node [draw,on chain=2] {\x};
    } 
\end{tikzpicture}\\

\begin{tikzpicture}[
      start chain=1 going right,start chain=2 going right,node distance=-0.15mm
    ]  
    \node [on chain=1] at (-1.5,-.4) {A:};
    \foreach \x in {\textbf{2},\textbf{3},\textbf{5},7,12,23,34,54,67,68,89} {
        \x, \node [draw,on chain=1] {\x};
    }    
    \node [on chain=2] at (-1.5,-1.0) {S:};
    \foreach \x in {11} {
        \x, \node [draw,on chain=2] {\x};
    } 
\end{tikzpicture}


\caption{Example of a normal IQS execution when retrieving the first element on a sorted array. Due to the inexistent pivots on S, every scan is made in lineal time.}\label{example1}
\end{figure}
